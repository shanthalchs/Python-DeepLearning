\documentclass[spanish,12pt]{report}
\usepackage{amsfonts}
\usepackage{graphicx}
\usepackage{graphicx,float}
\usepackage{amsmath, amsthm}
\usepackage{color}
\usepackage{url}
\usepackage[usenames,dvipsnames,svgnames,table]{xcolor}
\usepackage[utf8]{inputenc}
\usepackage[spanish]{babel}

\textheight=24cm \textwidth=18cm \topmargin=-2cm \oddsidemargin=-1.2cm
\setlength{\unitlength}{1 mm}
\parindent=0pt

\newcommand{\gfrac}[2]{\displaystyle{\frac{#1}{#2}}}

\begin{document}

\textsf{PROMiDAT Iberoamericano}

\textsf{Deep Learning con Python}

\textsf{Fundamentos Matemáticos del Deep Learning}

\begin{itemize}
{\color{Blue}
\item  Las tareas tienen fecha de entrega una semana después a la clase y deben ser entregadas antes del inicio de la clase siguiente.
\item  Cada día de atraso en implicará una pérdida de 10 puntos.
\item  Las tareas son estrictamente de carácter individual, tareas iguales se les asignará cero puntos.
\item  En nombre del archivo debe tener el siguiente formato: {\tt Tarea1\_nombre\_apellido.pdf}. Por ejemplo, si el nombre del estudiante es Luis Pérez: {\tt Tarea1\_luis\_perez.pdf}. Para la tarea número 2 sería: {\tt Tarea2\_luis\_perez.pdf}, y así sucesivamente.
\item Esta tarea tiene un valor de un 25\% respecto a la nota total del curso.
}
\end{itemize}

\begin{center}
{\color{Red} \textbf{\LARGE\sc Tarea Número 1}}
\end{center}

\begin{itemize}

\item {\bf{\color{Red} Ejercicio 1:}}  {\sf [5 puntos]}  Cree un tensor de 0D, 1D, 2D y 3D.

\item {\bf{\color{Red} Ejercicio 2:}}  {\sf [5 puntos]}  Imprima las dimensiones ({\tt shape}) de {\tt MNIST}.

\item {\bf{\color{Red} Ejercicio 3:}} {\sf [5 puntos]} ¿Qué tipo de datos contiene {\tt MNIST} ?.




\item {\bf{\color{Red} Ejercicio 4:}} {\sf [20 puntos]}	La operación {\tt relu()} es una operación que se aplica entrada por entrada de un vector, esta devuelve el máximo entre cada entrada del vector y 0 ({\tt relu(x) = max(x,0)}). Reprograme esta función “a mano” en {\tt Python} usando un {\tt for} y después pruébela con el siguiente vector {\tt x}. 
\begin{center}
    {\tt x = np.array[-1,3,-0.2,15]}
\end{center}
Imprima el resultado


\item {\bf{\color{Red} Ejercicio 5:}} {\sf [15 puntos]} 	Manualmente programe un “broadcast” para que el vector {\tt y = [4,1,1,5]} pueda ser sumado con el vector {\tt z = $\begin{bmatrix}
   3 & 6 & 7 & 1 \\
   2 & 6 & 2 & 9 \\
   1 & 1 & 1 & 3 \\
\end{bmatrix}$}.

Imprimalo.

\item {\bf{\color{Red} Ejercicio 6:}} {\sf [30 puntos]}	Calcule y muestre el procedimiento de las siguientes operaciónes de matrices:

\begin{itemize}
    \item $
\begin{pmatrix}
    8 & 4  \\
    1 & 9  \\
\end{pmatrix} \cdot
\begin{pmatrix}
    2 \\
    5 \\
\end{pmatrix}
$
\\
\\
    \item $
\begin{pmatrix}
    3 & 6  \\
    5 & 4  \\
\end{pmatrix} \cdot
\begin{pmatrix}
    1 & 3 \\
    7 & 2 \\
\end{pmatrix}
$
\\
\\
    \item $
\begin{pmatrix}
    8 & 5 & 1 \\
    1 & 2 & 4 \\
    7 & 3 & 2
\end{pmatrix} \cdot
\begin{pmatrix}
    1 & 3 \\
    4 & 5 \\
    2 & 3
\end{pmatrix}
$
\end{itemize}

\item {\bf{\color{Red} Ejercicio 7:}} {\sf [15 puntos]} Calcule las siguientes derivadas. Muestre el procedimiento:

\begin{itemize}
    \item[a)] $f(x) = 4x$
    \item[b)] $f(x) = 6x^2 + 12x$
    \item[c)] $f(x) = \sqrt{x}$
    \item[d)] $f(x) = \cos(x)$
     \item[e)] $f(x) = 3 + \ln(x)$
\end{itemize}

\item {\bf{\color{Red} Ejercicio 8:}} {\sf [5 puntos]} Calcule el gradiente de la siguiente función

$$
f(x,y,z) = 3x^4 - 6xyz + 3yz
$$

\end{itemize}


\vspace{2cm}

\begin{center}
\includegraphics[height=4cm]{logo.jpg}
\end{center}

\end{document}

