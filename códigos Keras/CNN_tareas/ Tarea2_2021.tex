\documentclass[spanish,12pt]{report}
\usepackage{amsfonts}
\usepackage{graphicx}
\usepackage{graphicx,float}
\usepackage{amsmath, amsthm}
\usepackage{color}
\usepackage[usenames,dvipsnames,svgnames,table]{xcolor}
\usepackage[utf8]{inputenc}
\usepackage[spanish]{babel}
\usepackage{hyperref}

\textheight=24cm \textwidth=18cm \topmargin=-2cm \oddsidemargin=-1.2cm
\setlength{\unitlength}{1 mm}
\parindent=0pt

\newcommand{\gsum}[2]{\mbox{$ \displaystyle{\sum_{#1}^{#2}} $}}

\begin{document}

\textsf{PROMiDAT Iberoamericano}

\textsf{Deep Learning}

\textsf{Instrucciones:}

\begin{itemize}
{\color{Blue}
\item  Las tareas tienen como fecha de entrega una semana después a la clase y deben ser entregadas antes del inicio de la clase siguiente.
\item  Cada día de atraso implicará una pérdida de 10 puntos.
\item  Las tareas son estrictamente de carácter individual, tareas iguales se les asignará cero puntos.
\item  El nombre del archivo debe tener el siguiente formato: {\tt Tarea1\_nombre\_apellido.pdf}. Por ejemplo, si el nombre del estudiante es Luis Pérez: {\tt Tarea1\_luis\_perez.pdf}. Para la tarea número 2 sería: {\tt Tarea2\_luis\_perez.pdf}, y así sucesivamente.
\item  El puntaje de cada pregunta se indica en su encabezado.
\item Esta tarea tiene un valor de un 25\% respecto a la nota total del curso.
}
\end{itemize}

\begin{center}
{\color{Green} \textbf{\LARGE\sc Tarea Número 1}}
\end{center}

\begin{itemize}

\item {\bf{\color{Red} Ejercicio 1:}} {\sf [40 puntos]} Este conjunto de datos  tiene 10 clases que contienen 600 imágenes cada una. Hay 500 imágenes de entrenamiento y 100 imágenes de prueba por clase. Cada imagen viene con una etiqueta "fina" (la clase a la que pertenece). El objetivo del conjunto de datos es predecir de forma diagnóstica si un paciente tiene diabetes o no, basándose en determinadas medidas de diagnóstico incluidas en el conjunto de datos.

El conjunto de datos tiene 390 filas y 16 columnas.

\begin{itemize}

\item {\tt  Clase 0}: Manzana. 
\item {\tt  Clase 1}: Pez. 
\item {\tt  Clase 2}: Bebé.
\item {\tt  Clase 3}: Oso.
\item {\tt  Clase 4}: Castor.
\item {\tt  Clase 5}: Cama.
\item {\tt  Clase 6}: Abeja.
\item {\tt  Clase 7}: Escarabajo.
\item {\tt  Clase 8}: Bicicleta.
\item {\tt  Clase 9}: Botella.



\end{itemize}

Realice lo siguiente:

\begin{enumerate}
\item Cargue las tablas de datos  \texttt{imagenes\_x\_train.npy}, \texttt{imagenes\_y\_train.npy}, \texttt{imagenes\_x\_test.npy} y  \texttt{imagenes\_y\_test.npy}  en {\tt Python}. {\bf }.

\item Pase las tablas de las categórias (\texttt{imagenes\_y\_train.npy} y \texttt{imagenes\_y\_test.npy}) a la forma {\bf one-hot-encoding} 

\item Imprima una imagen de la tabla \texttt{imagenes\_x\_train.npy}

\item Normalice las imágenes de las tablas .   

\item Haga el modelo usando la función de activación {\tt relu} y con las capas que considere necesarias. Utilice la función de optimización {\tt sgd}, la función de costo {\tt categorical\_crossentropy} y las métricas {\tt acc} y {\tt mse}.

\item Haga un resumen del modelo.

\item Haga una predicción usando 32 bloques y 5 {\tt epochs}, use {\tt x\_test} para validar.

\item Genere la matriz de confusión  y haga una visualización de esta.

\item Calcule la precisión global. Interprete
la calidad de los resultados.

\item Comente los resultados.

\end{enumerate}



\item {\bf{\color{Red} Ejercicio 2:}} {\sf [40 puntos]} En este ejercicio vamos a usar las tablas de datos {\tt ZipDataTestCod.csv} {\tt ZipDataTrainCod.csv}, estos son archivos en csv, donde la primera columna guarda la información de tipo de digito es: un numero del 0 al 9, mientras que las otras 256 columnas guardan la información de cada pixel, para una imagen de 16x16 de un solo canal. La información de los csv son de tipo {\tt int}.


Para esto realice lo siguiente:

\begin{enumerate}
\item Cargue las tablas de datos  {\tt ZipDataTestCod.csv} {\tt ZipDataTrainCod.csv} en {\tt Python}. {\bf }.

\item Cree las tablas de las categórias ({\tt Y\_train}  y {\tt Y\_test}) y paselas a la forma {\bf one-hot-encoding} 

\item Cree las tablas de las imágenes ({\tt X\_train}  y {\tt X\_test}) y seguidamente normalice dichas tablas.

\item Imprima una imagen de la tabla \texttt{X\_train}

\item Haga el modelo usando la función de activación {\tt relu} y con las capas que considere necesarias. Utilice la función de optimización {\tt adam}, la función de costo {\tt categorical\_crossentropy} y las métricas {\tt acc} y {\tt mse}.

\item Haga un resumen del modelo.

\item Haga una predicción usando 86 bloques y 10 {\tt epochs}, use {\tt x\_test} para validar.

\item Genere la matriz de confusión  y haga una visualización de esta.

\item Calcule la precisión global. Interprete
la calidad de los resultados.

\item Comentes los resultados.

\end{enumerate}



\begin{center}
\includegraphics[height=8cm]{Logo.jpg}
\end{center}

\end{itemize}
\end{document}


